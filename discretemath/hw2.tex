\documentclass[11pt,fleqn]{article}
\usepackage{latexsym,epsf,epsfig}
\usepackage{amsmath,amsthm}
\begin{document}
\newcommand{\mbf}[1]{\mbox{{\bfseries #1}}}
\newcommand{\N}{\mbf{N}}
\newtheorem{theorem}{Theorem}
\newtheorem{lemma}{Lemma}
\def\floor#1{\lfloor #1 \rfloor}

\section*{CS 70 homework 2 solutions}

Your full name: PUT YOUR NAME HERE
\newline
Your login name: PUT YOUR LOGIN NAME HERE
\newline
Homework 2
\newline
Your section number: PUT YOUR SECTION NUMBER HERE
\newline
Your list of partners: LIST YOUR PARTNERS HERE
\newline



\begin{enumerate}

\item
Prove the proposition
\begin{displaymath}
(P \implies (Q \implies (P \land Q))) 
\implies (\lnot (P \land Q) \implies (P \implies \lnot Q))
\end{displaymath}

YOUR ANSWER GOES HERE.

\item
Prove that $2^n < n!$ for all integers $n \ge 4$.

YOUR ANSWER GOES HERE.

\item
\begin{enumerate}
\item
Find a formula that relates the values of {\tt n1}, {\tt n2}, and {\tt so-far}
at the start of each call to {\tt product-helper} to the product of the
{\tt a1} and {\tt a2}, the original arguments of the {\tt product} procedure.

YOUR ANSWER GOES HERE.
\item
Using your invariant relation, prove that {\tt product} returns the product
of its two nonnegative integer arguments.

YOUR ANSWER GOES HERE.
\end{enumerate}

\item
Prove that {\tt reverse} works.
You will need some formal definitions:
if $L = (a_0\ a_1\ a_2\ ...)$, then 
\begin{eqnarray*}
{\tt(car\ L)} = a_0,
{\tt (cdr\ L)} = (a_1\ a_2\ ...), 
{\tt (cons\ x\ L)} = (x\ a_0\ a_1\ a_2\ ...).
\end{eqnarray*}

You then are to prove that if
$L = (x_0\ x_1\ x_2\ ...\ x_N)$, {\tt (reverse L)} returns $(x_N\ x_{N-1}\ ...\ x_1\ x_0)$.

YOUR ANSWER GOES HERE.

\item
\begin{enumerate}
\item
Prove that any Sprouts game consists of a finite number of moves
before someone loses.

YOUR ANSWER GOES HERE.
\item
Give as good an upper bound as you can on the worst-case number of
moves in a Sprouts game that starts with $n$ dots, and prove your answer.

YOUR ANSWER GOES HERE.
\end{enumerate}


\item
Prove by induction that $\sum_{i=1}^n \floor{i/2} = \floor{n^2/4}.$
(Recall that $\floor{x}$ is the largest integer less than or equal to $x$.)

YOUR ANSWER GOES HERE.

\item
Working at the local pizza parlor, I have a stack of unbaked
pizza doughs.
For a most pleasing presentation, I wish
to arrange them in order of size, with the largest pizza on the bottom.
I'm able to place my spatula under one of the pizzas and flip over
the whole stack above the spatula (reversing their order).

This is the only move I can do to change the order of the stack;
however, I am willing to keep repeating this move until I get the stack
in order.
Is it always possible for me to get the pizzas in order?
Prove your answer.

YOUR ANSWER GOES HERE.

\item
Assign a grade of A (correct) or F (failure) to
each of the following proofs.  If you give a F, please
explain exactly everything that is wrong with the structure or the reasoning in the
{\it proof}. You should justify all your
answers (remember, saying that the claim is false is {\it not} 
a justification).

\begin{enumerate}
\item
\begin{theorem}
For every $n \in \N$, $n^2+n$ is odd.
\end{theorem}
\begin{proof}
The proof will be by induction.\\
\emph{Base case:} The natural number 1 is odd.\\
\emph{Inductive step:} Suppose $k \in \N$ and $k^2 + k$ is odd.  Then,
$${(k+1)^2 + (k+1) = k^2 + 2k + 1 + k +1 = (k^2 + k) + (2k+2) }$$
is the sum of an odd and an even integer.  Therefore, $(k+1)^2 + (k+1)$
is odd.  By the Principle of Mathematical Induction, the property that
$n^2+n$ is odd is true for all natural numbers $n$. 
\end{proof}

YOUR ANSWER GOES HERE.

\item
\begin{theorem}
For all $x,n\in \N$, if $nx=0$ and $n>0$, then $x=0$.
\end{theorem}
\begin{proof}
The proof will be by induction.\\
\emph{Base case:} If $n=1$, then the equation $nx=0$
implies $x=0$, since $nx=1\cdot x=x$ in this case.\\
\emph{Induction step:} Fix $k>0$, and assume that $kx=0$ implies $x=0$.
Suppose that $(k+1)x=0$.
Note that $(k+1)x=kx+x$, hence we can conclude that $kx+x=0$,
or in other words, $kx=-x$.
Now there are two cases:
\begin{description}
\item[Case 1:] $x=0$. In this case, $kx=-x=-0=0$, so $kx=0$.
Consequently, the inductive hypothesis tells us that $x=0$.
\item[Case 2:] $x>0$. In this case, $-x<0$ (since $x>0$).
At the same time, $kx\ge 0$ (since $k,x \ge 0$).
But this is impossible, since we know $kx=-x$.
We have a contradiction, and therefore Case 2 cannot happen.
\end{description}
In either case, we can conclude that $x=0$.
This completes the proof of the induction step.
\end{proof}

YOUR ANSWER GOES HERE.

\item
\begin{theorem}
For all $x,y,n\in \N$, if $\max(x,y)=n$, then $x=y$.
\end{theorem}
\begin{proof}
The proof will be by induction.\\
\emph{Base case:} Suppose that $n=0$.  If $\max(x,y)=0$
and $x,y\in N$, then $x=0$ and $y=0$, hence $x=y$.\\
\emph{Induction step:} Assume that, whenever
we have $\max(x,y)=k$, then $x=y$ must follow.
Next suppose $x,y$ are such that $\max(x,y)=k+1$.
Then it follows that $\max(x-1,y-1)=k$, so by the inductive
hypothesis, $x-1=y-1$.
In this case, we have $x=y$, completing the induction step.
\end{proof}

YOUR ANSWER GOES HERE.

\item
\begin{theorem}
$\forall n \in \N.\;\; n^2 \leq n$.
\end{theorem}
\begin{proof}
The proof will be by induction.\\
\emph{Base case:}
When $n=0$, the statement is $0^2 \leq 0$ which is true. \\
\emph{Induction step:}
Now suppose that $k \in \N$, and $k^2 \leq k$.
We need to show that $$(k+1)^2 \leq k+1$$
Working backwards we see that:
\begin{eqnarray*}
(k+1)^2 &\leq& k+1 \\
k^2 + 2k + 1 &\leq& k + 1 \\
k^2 + 2k &\leq& k \\
k^2 &\leq& k
\end{eqnarray*}
So we get back to our original hypothesis which is assumed to be true.
Hence, for every $n \in \N$ we know that $n^2 \leq n$.
\end{proof}

YOUR ANSWER GOES HERE.

\end{enumerate}
\end{enumerate}
\end{document}


