\documentclass{article}
\usepackage[utf8]{inputenc}
% set font encoding for PDFLaTeX, XeLaTeX, or LuaTeX
\usepackage{ifxetex,ifluatex}
\usepackage{amsmath}
\usepackage{amsfonts}
\usepackage{proof}
\if\ifxetex T\else\ifluatex T\else F\fi\fi T%
  \usepackage{fontspec}
\else
  \usepackage[T1]{fontenc}
  \usepackage[utf8]{inputenc}
  \usepackage{lmodern}
\fi

%\usepackage{hyperref}

\title{cs70-hw1}
\author{czpt101 }
\date{October 2021}

\begin{document}

\maketitle

\newcommand{\mbf}[1]{\mbox{{\bfseries #1}}}
\newcommand{\N}{\mbf{N}}
\newcommand{\Z}{\mbf{Z}}

\section*{CS 70 homework 1 solutions}

Your full name: Galina Khayut
\newline
Your login name: czpt101
\newline
Homework 1

\begin{enumerate}

\item
For each of the following, define proposition
symbols for each simple proposition in the argument (for example, $P$ =
``I will ace this homework''). Then write out the logical form of
the argument. If the argument form corresponds to a known inference
rule, say which it is. If not, show that the proof is correct using
truth tables.
\begin{enumerate}
\item  I will ace this homework and I will have fun doing it.
Therefore, I will ace this homework.

YOUR ANSWER GOES HERE.

P - I will ace this homework (true) \\
Q - I will have fun doing it (true) \\
Implication: I will ace this homework (true) \\ 
P AND Q infers P OR \[ \infer {P}{P \wedge Q} \]

\item It is hotter than 100 degrees today or the pollution is
dangerous. It is less than 100 degrees today. Therefore, the pollution
is dangerous.

YOUR ANSWER GOES HERE.

P - it is hotter than 100 today \\
Q - the pollution is dangerous \\
Proposition: if it's less than 100, the pollution is dangerous (converse) \\
\[ \neg P \to Q \] 

\item Tina will join a startup next year. Therefore,
Tina will join a startup next year or she will be unemployed.

YOUR ANSWER GOES HERE.

P - Tina will join a startup next year \\
Q - Tina will be unemployed \\
Proposition: If Tina joins a startup next year, she will not be unemployed \\
\[ P \to \neg Q \] \\
Proposition: If Tina is unemployed, she didn't join the startup (converse)\\
\[ Q \to \neg P \]

\item If I work all night on this homework, I will answer all the
exercises. If I answer all the exercises, I will understand the
material. Therefore, if I work all night on this homework, I will
understand the material.

YOUR ANSWER GOES HERE.

P - I will work all night \\
Q - I will answer all the exercises \\
\[ P \to Q \] \\
P1 - I will understand the material \\
\[ P \to Q \wedge Q \to P1, \; P \to P1 \; (transitivity). \] \\

\end{enumerate}


\item
Recall that $\N=\{0,1,\ldots\}$ denotes the set of natural numbers,
and $\Z=\{\ldots,-1,0,1,\ldots\}$ denotes the set of integers.

\begin{enumerate}

\item Define $P(n)$ by
\[ P(n) = \forall m \in \N . \; m<n \to
\neg (\exists k \in \N . \; n=m*k \; \wedge \; k<n). \]
Concisely, for which numbers $n\in\N$ is $P(n)$ true?
\newline
YOUR ANSWER GOES HERE.
\newline \newline
$P(n)$ is defined as for all values of $m \in N$ s.t. $(m<n)$ \\

First, let's find all n that are a counter-example.

For all values of  $m$  st $(m<n)$, there EXISTS a $k$ st \\
$(n=m \cdot k \; AND \; k<n \; AND \; (m, n, k) \; \in \N)$,
\\  OR \\
\[ P(n)= \forall m \; st  \; (m<n) 
\exists k \; st \; (n=m \cdot k) \wedge (k<n) \wedge \; (m, n, k \in\N) \] 

The proposition will hold true for all values of $m$ because there will always be an $n$ that is a product of $m$ and another natural integer $k<n$ if $n$ is non-prime. 

Therefore, the original proposition will hold true $IFF$ n IS PRIME.


\item Rewrite the following in a way that
removes all negations (``$\neg$, $\ne$'') but remains equivalent.
\[ \forall i . \; \neg \forall j . \;
\neg \exists k . \; 
(\neg \exists \ell . \; f(i,j) \ne g(k,\ell)). \]
\newline
YOUR ANSWER GOES HERE.
\newline
Proposition:

For all values of $i$  there exist some values of $j$ and there are no values of $k$ and $l$ such that $f(i,j)=g(k,l)$ 

\[ P(i,j,k,l) = \forall i  \; \exists j  \;
\forall k, l \; f(i,j)=g(k,l) \] 	


\item Prove or disprove:
$\forall m \in \Z . \; \exists n \in \Z . \; m \ge n$.

YOUR ANSWER GOES HERE.
\newline
For all integer values of m there exists an integer $n$ such that $m>=n$.

Proof by example:

Let $n$ and $m$ be equal or consecutive integers. Thus, $m>=n$ holds TRUE.
\end{enumerate}
\newpage

\item
Alice and Bob are playing a game of chess,
with Alice to move first.
If $x_1,\dots,x_n$ represents a sequence of possible moves
(i.e., first Alice will make move $x_1$, then Bob will make move $x_2$,
and so on),
we let $W(x_1,\dots,x_n)$ denote the proposition that,
after this sequence of moves is completed,
Bob is checkmated.

\begin{enumerate}
\item State using quantifier notation the proposition that Alice
can force a checkmate on her second move, no matter how Bob plays.
\newline \newline
YOUR ANSWER GOES HERE.
\newline \newline
Given that Bob always loses if Alice moves first, the original proposition could be rephrased as "Bob will lose IFF Alice makes the first move". In other words, if Bob's moves are denoted by even numbers, Bob will always lose."
An even number x can be represented as $x=2k$ where $k$ is a positive integer.

Proposition:
$W(x)$= Bob loses a chess game \\
\[ W(x_1, \dots, x_n) = \forall x \in \N , \; x=2k . \; (k \in \N \wedge k \ge 1) \]


\item Alice has many possibilities to choose from on her first move,
and wants to find one that lets her force a checkmate on her second move.
State using quantifier notation the proposition that $x_1$
is \emph{not} such a move.
\newline \newline
YOUR ANSWER GOES HERE.
\newline \newline
Proposition: 

$P(x)$: Bob loses on the third move $x_3$ \\
$W(x)$: There exists an $x_1$ s.t. $P(x)$ is TRUE \\
or,
\[ W(x): \forall x \in (\N,  x_1, \dots, x_n) \; \exists x_1 \; s.t. \; W(x_1) \to P(x_3)) \]

To negate the statement:
\[ W(x)\prime = \forall x \in \N, x_1 \dots x_n \; \neg \exists x_1 \; s.t. \; W(x_1) \to P(x_3) \]
\end{enumerate}
\newpage

\item
Joan is either a knight or a knave.
Knights always tell the truth, and only the truth;
knaves always tell falsehoods, and only falsehoods.
Someone asks Joan, ``Are you a knight?''  She replies,
``If I am a knight then I'll eat my hat.''
\begin{enumerate}
\item Must Joan eat her hat?
\newline \newline
YOUR ANSWER GOES HERE.
\newline \newline
The statement 
"If Joan is a knight, she will eat her hat" evaluates to TRUE because knights tell the truth, so she'll have to eat her hat.\\

If Joan is a knave, the equal statement would be \\
"If Joan doesn't eat her hat, she is not a knight." \\
But if knaves always lie, then the promise to eat her hat if she were a knight is FALSE. So, if she doesn't eat her hat we can deduce that she is indeed a knave, but the equivalence statement is FALSE, so to convince us she is a knight and not a knave, she'll have to eat her hat either way.
She kinda brought it on herself.

\item Let's set this up as problem in propositional logic.
Introduce the following propositions:
\begin{align}
P &= && \text{``Joan is a knight''}\\
Q &= && \text{``Joan will eat her hat''}.
\end{align}

Translate what we're given into propositional logic,
i.e., re-write the premises in terms of these propositions.

YOUR ANSWER GOES HERE.
\newline \newline
\[ P \to Q \] 
or, its equivalent \\
\[ \neg Q \to \neg P \]

\item Using proof by enumeration,
prove that your answer from part (1) follows from the premises
you wrote in part (2).
(No inference rules allowed.)

YOUR ANSWER GOES HERE.
\newline 
\begin{enumerate}
    \item Joan is a knight (P) \\
    Proposition: if Joan is a knight = TRUE, she'll eat her hat 
    \[ P \to Q \; = TRUE \] 
    because knights tell the truth.
    \item Joan is a knave $(\neg P)$ \\
    The equivalence statement: 
    If Joan doesn't eat her hat, she's a knave
    \[\neg Q \to \neg P \; = \; FALSE \] because knaves lie.\\
    The final proposition:
    \begin{align}
        P &=&& \text{''Joan is a knight''} \\
        \neg P &=&& \text{''Joan is a knave''} \\
        Q &=&& \text{''Joan will eat her hat''} \\
        (P \lor \neg P) \to Q.
    \end{align}
    
\end{enumerate}
    
\end{enumerate}
\newpage

\item
For each claim below,
prove or disprove the claim.

\begin{enumerate}
\item Every positive integer can be expressed as the sum of two perfect squares. 
(A perfect square is the square of an integer. 0 may be used in the sum.)

YOUR ANSWER GOES HERE.
\newline \newline
Statement: 
Show that there exist $m,n$ s.t. for all values of $k$, $k$ can be expressed as $m^2+n^2$:
\[ P(k)= \forall k \in \N \; \exists m,n \in \Z \; s.t. \; k=m^2+n^2 \]
Proof: \\
Logically equivalent statement would be the negation: \\
There ISN'T a positive integer that cannot be expressed as the sum of two perfect squares. \\
The statement is FALSE by counterexample: \\
Let x be a positive integer that is represented as a sum of a perfect square and a prime number: 1+5, 4+7 etc. A prime number cannot be a perfect square by definition (unless you count 1 but that's another can of worms)\\

The correct statement would be:\\
There ISN'T a positive integer that cannot be expressed as a sum of two real numbers squared (or "imperfect squares" of rational or irrational numbers).
\[ \neg P(k)= \neg \exists k \in \N \;  \neg \exists m,n \in \mathbb{R} \; s.t. \; k \ne m^2+n^2 \]
\newpage

\item For all rational numbers $a$ and $b$, $a^b$ is also rational.

YOUR ANSWER GOES HERE.
\newline \newline
Statement: 
\[ P(a,b)= \forall a,b \; \in \mathbb{Q}, \; a^b \; \in \mathbb{Q} \]

Rational numbers $a,b$ can be expressed as $x/y$ and $w/z$ where $x,y,w,z$ are integers. Let's assume that $a^b$ can also be expressed as a ratio of two integers:\\
\[ (x/y)^{w/z}=x^{w/z}/y^{w/z} \]
Prove: 
\[ P(x,y,w,z)= \forall x,y,w,z \in \Z, \; x^{w/z}, y^{w/z} \; \in \Z\]
The statement is FALSE, since there exist values of $w,z$ such that $\sqrt[z]{x^w}$ is irrational, for example, with $x=2$ and $w$ and $z$ being consecutive integers. Counterexample: 
\[ \sqrt[2]{x^1} = \sqrt{2}\]
\end{enumerate}
\end{enumerate}
\end{document}
\maketitle
\end{document}











